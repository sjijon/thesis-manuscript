% !TEX root = ../ThesisManuscript_SJ.tex
%%
%% ABBREVIATIONS & Glossary
%%_______________________________________________
\chapter*{Acronyms and abbreviations}
\addcontentsline{toc}{chapter}{Acronyms and abbreviations}
\markboth{ACCRONYMS AND ABBREVIATIONS}{}

%	\bf & \\

\begin{table}[H]
\begin{tabular}{ll}
	\multicolumn{2}{l}{\large \bf Acronyms}\\
	\\
	\bf AIDS		& Acquired immunodeficiency syndrome\\
	\bf ART 		& Anti-retroviral therapy \\
	\bf CDC 		& Centers for disease control and prevention\\
	\bf CI 		& Confidence interval \\
	\bf DFS 		& Disease-free state \\
	\bf ECDC 		& European centre for disease prevention and control\\
	\bf ES 		& Endemic state \\
	\bf GVAP 		& Global vaccine action plan\\
	\bf HIV 		& Human immunodeficiency virus \\
	\bf IdF 		& \^Ile-de-France \\
	\bf MMR 		& Measles-Mumps-Rubella vaccine\\
	\bf MMWR	& Morbidity and mortality weekly report from the CDC\\
	\bf MSM 		& Men who have sex with men \\
	\bf ODE 		& Ordinary differential equations \\
%	\bf PAHO 		& Pan American health organization\\
	\bf PrEP 		& Pre-exposure prophylaxis \\
	\bf STI(s)		& Sexually-transmitted infection(s) \\
	\bf SDG		& Sustainable Development Goals \\
	\bf TDF/FTC 	& Tenofovir disoproxil fumarate/Emtricitabine \\
	\bf UK 		& United Kingdom\\
	\bf US		& United States\\
	\bf UNAIDS 	& Joint united nations program on HIV/AIDS \\
	\bf WHO		& World health organization \\
\end{tabular}
\end{table}

\vfill
\newpage

\begin{table}[H]
\begin{tabular}{ll}
	\multicolumn{2}{l}{\large \bf Abbreviations}\\
	\\
	\bf Eq(s). 		& Equation(s) \\
	\bf Fig(s). 		& Figure(s) \\
	\bf Ref(s).		& Reference(s) \\
%	\bf Resp.		& Respectively \\
\end{tabular}
\end{table}


%%_______________________________________________
\chapter*{Glossary}
\addcontentsline{toc}{chapter}{Glossary}
\markboth{GLOSSARY}{}

%\section*{Epidemiology}

\begin{description} 
	\item[Compliance.] (Also called adherence). Behavior that follows the recommendations of a physician or other healthcare provider, or investigator in a research project~\cite[]{Porta2014}.
%	\item[Control (of an epidemic).] Programs and policies aiming to epidemic elimination or the reduction of new infections~\cite[]{Porta2014}. Here, we use control to refer to the reduction in the number of new infections resulting from these programs. 
	\item[Effectiveness (of prevention).] Relative reduction in the number of infections, resulting from studies carried out under less than perfectly controlled conditions (more similar to typical behavior)~\cite[]{CDC_Epidemiology}.
	\item[Efficacy (of prevention).] Relative reduction in the number of infections, resulting from studies carried out under ideal conditions (a.e., clinical trials)~\cite[]{CDC_Epidemiology}.
%	\item[Elimination (of an epidemic).] Reduction of disease transmission to a predetermined very low level~\cite[]{Porta2014}.
	\item[Eradication (of a disease).] Termination of all transmission of infection by extermination of the infections agent through surveillance and globally coordinated efforts~\cite[]{Porta2014}. Once the eradication status is achieved, control interventions are no longer necessary.
	\item[Herd immunity]Immunity of a group or a community~\cite[]{Porta2014}.
	\item[Incidence rate (force of infection, person-time incidence rate).] Theoretical measure of the number of new cases that occur per unit of population-time. Mathematically defined as~\cite[]{Porta2014}
	\[
	\lim_{\Delta t \to 0}\frac{\text{Probability that a person well at time } t \text{ gets infected in the time interval } [t,t+\Delta]}{\Delta t}
	\]
	which can be estimated by
	\[
	\frac{\text{Number of new cases observed in the time interval } [t,t+\Delta]}{\text{Number of person-time units of experience observed in the time interval } [t,t+\Delta]}.
	\]
	\item[Incubation period.] The interval of time between the infection and the first symptoms of disease~\cite[]{Porta2014}.
	\item[Prevalence.] Proportion of infected individuals at a specified time or period~\cite[]{Porta2014}.
	\item[Prevention.] Actions that prevent disease transmission and/or infection~\cite[]{Porta2014}.
	\item[Prophylaxis.] Preventive healthcare~\cite[]{Porta2014}.
%	\item[Reproduction number.] Number of secondary cases that an infected individual induces in a fully susceptible population.
%	\item[Utility.] (also called \emph{payoff}).
\end{description}



%\newpage
%\thispagestyle{plain}
%\mbox{}