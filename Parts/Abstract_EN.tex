%% !TEX root = ../ThesisManuscript_SJ.tex
%%
%%
%%	ABSTRACT
%%_______________________________________________

\chapter*{\centering \Large \vspace{-3cm}
		Abstract
		\vspace{-0.5cm}}
\addcontentsline{toc}{chapter}{Abstract}
\markboth{ABSTRACT}{}

\small

Despite the current availability of effective preventive methods, controlling epidemics of preventable infectious diseases remains a key public health challenge. When facing an ongoing epidemic, individuals may decide to use prevention, or else to get treated in the case of acquiring the infection. Whereas treatment is generally well accepted by infected individuals, the acceptability of prevention may vary between individuals, which may lead to preventive behaviors that differ from the recommendations of the public health authorities. 

My doctoral research concerns the mathematical modeling of infectious diseases transmission, taking into account the individuals' {\it prevention versus treatment dilemma}, the decision-making on whether or not to adopt prevention to avoid infection during an ongoing epidemic, in a context where efficient treatment is available. We aim to determine whether and under what conditions the voluntary adoption of prevention could avert an epidemic. 

%\paragraph{Methods.} 

We propose a mathematical model that combines the disease transmission at the population level and the decision-making at the individual level. We model disease transmission using a compartmental model given by a system of ordinary differential equations. For the individual-level decision-making, we rely on a game-theoretic approach, which assumes that individuals solve the prevention versus treatment dilemma by choosing the strategy ---to use prevention or not--- that benefits them the most. The decision-making depends on the individuals' perception of their risk of infection, as well as on their perception of the relative cost of prevention versus treatment, which includes monetary and/or non-monetary aspects such as price, reimbursement policies, accessibility, social stigma, disease morbidity, undesired secondary effects, etc. 

We explore two cases of the dilemma of prevention versus treatment. First, we address voluntary vaccination in the context of preventable and treatable childhood infectious diseases. In particular, we apply our methods and findings to the epidemiology of measles. 
Second, we study the voluntary adoption of pre-exposure prophylaxis to avoid HIV acquisition by the individuals who are most at risk of infection. In particular, we analyze the HIV epidemiology among one of the populations most at risk in France: men who have sex with men in the Paris region.


%\paragraph{Findings.} 

We obtain the probability that an individual voluntarily adopts prevention, as a function of the parameters of the prevention method (namely, effectiveness and cost). Our results suggest that epidemic elimination (i.e., the absence of new infections) is possible, provided that preventive methods are highly effective and that individuals perceive the relative cost of prevention versus treatment to be low.
%
However, epidemic elimination may be only temporary. Once the epidemic is averted, there is no long-term motivation to adopt prevention based in the individual's perception of the risk of infection. An important decrease in the number of infections may reveal  less disease burden to individuals, who, in turn, perceive less benefit from prevention. In other words, epidemic elimination may induce a higher cost for prevention, as perceived by individuals. Hence, active efforts to maintain the cost of prevention low are required to preserve epidemic elimination in the long run.\\

\paragraph{Keywords.} Behavioral epidemiology; Voluntary prevention; Epidemic elimination; Game theory; Compartmental model
