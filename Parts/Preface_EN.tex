%% !TEX root = ../ThesisManuscript_SJ.tex
%%
%%_______________________________________________
\chapter*{Preface} 
\addcontentsline{toc}{chapter}{Preface}
\markboth{PREFACE}{}

This thesis was prepared at the Pierre Louis Institute of Epidemiology and Public Health (IPLESP -- join unit of research and health UMRS 1136 of Sorbonne University (SU) \& Inserm), within the team of Communicable diseases surveillance and modelling.

The first three years of PhD were funded through a doctoral contract from the French ministry of higher education and research, awarded through the Doctoral network in public health (RDSP), coordinated by the School of advanced studies in public health (EHESP). The fourth year of the PhD was funded by a research grant from the French research agency on AIDS and viral hepatitis (ANRS). During the last year of thesis, the PhD candidate held a research and teaching adjunct contract (ATER) at the Biology Faculty of SU and the Laboratory of Computational and Quantitative  Biology (LCQB -- join unit of research UMR 7238 SU \& CNRS).

%\subsubsection*{Manuscript description}

This thesis is devoted to public health issues in infectious disease epidemiology and individual behaviors regarding prevention,  from the perspective of mathematical modeling. The approach is interdisciplinary, covering expertise from health-care providers about scientific communication, clinical trials processes and the implementation of public health interventions, next to epidemiological models of disease transmission and economics tools to model decision-making and data on individual behavior.

The completion of this thesis is contemporaneous with three major events of epidemiological interest. First, the resurgence of measles due to decline in vaccine coverage, after decades of successful mitigation. Second, the authorization and rollout of pre-exposure prophylaxis against HIV infection for at-risk individuals. Third, the emergence of the SARS-CoV-2 pandemic, which yielded broad implementation of non-pharmaceutical prevention interventions as well as the rollout of mass vaccination vaccination programs. These events highlight the pertinence and urgency to study epidemic control through prevention interventions, accounting for the active participation of individuals. 

This manuscript is structured as follows:~\chapref{Introduction} presents a general introduction to the subject of voluntary prevention of infectious diseases in a context where effective treatment exists, as well as the conceptual framework and modeling approaches that we used to study the subject. As part of my doctoral research, two scientific articles were 
%published in peer-reviewed journals
produced~\cite[]{Jijon2017,Jijon2021}, which constitute the two main parts of this manuscript;~\chapref{Vaccine} and~\chapref{PrEP} present these two articles, respectively. \hyperlink{Vaccine}{Chapter~\ref*{Vaccine}} focuses on the modeling of voluntary vaccination against treatable childhood infectious diseases. \hyperlink{PrEP}{Chapter~\ref*{PrEP}} focuses on the voluntary use of pre-exposure prophylaxis to avoid HIV infection among the population of men who have sex with men. Each of these two chapters includes a specific introduction, where the epidemiology of the public health issue that motivates our work is presented, as well as some additional material that was not included in the articles. \hyperlink{Discussion}{Chapter~\ref*{Discussion}} presents a general discussion and conclusions. 

An interdisciplinary note was written in the context of the RDSP, and is available in \hyperlink{NID}{Appendix~\ref*{NID}}. This thesis is written in English; a detailed summary in French is provided in \hyperlink{chapter:ResumeDetailleFR}{Appendix~\ref*{chapter:ResumeDetailleFR}}.