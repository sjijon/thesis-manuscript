% !TEX root = ../ThesisManuscript_SJ.tex
%
% 	GENERAL DISCUSSION
%___________________________________________________________________________________
\chapter{General discussion}
\label{Discussion}
\markboth{GENERAL DISCUSSION}{}

\section{Summary}

\subsection{Reaching epidemic elimination through voluntary adoption of prevention}

We modeled the individual response to an epidemic threat through the resolution of a prevention-versus-treatment dilemma, and studied its impact on epidemic dynamics. We assumed that the individual willingness (respectively, refusal) to adopt preventive methods relies on the perception that the strategy of preventing the infection is more (respectively, less) beneficial than facing the risk of being infected, which could lead to acquiring the disease, and consequently being treated. To model the individual-level risk assessment, we assumed that individuals acknowledge, directly or indirectly, some epidemiological data (e.g., the disease prevalence \cite[]{Jijon2017} and the incidence rate \cite[]{Jijon2021}) provided by, for instance, public health authorities, through communication campaigns. 

We used our approach to address two public health issues. First, we studied voluntary vaccination against treatable childhood infectious diseases in a context where efficient, yet imperfect vaccines are available (cf.~\autoref{Vaccine} and \cite{Jijon2017}). The results of the vaccination model were obtained analytically, and provided important insights into the system properties, thus constituting a theoretical guide for the programming choices and the interpretation of the results of the second project that involved numerical implementation. Second, we studied the voluntary use of PrEP to avoid HIV infection, among the population of MSM (cf.~\autoref{PrEP} and \cite{Jijon2021}). For the HIV model, we accounted for population heterogeneity regarding the risk of infection, namely, due to heterogeneity in sexual behaviors. We considered that the high-risk population drives the epidemic, and thus becomes the target population for PrEP implementation policies, unlike childhood vaccination, which is recommended for the vast majority of newborns and young children. 

Our model's main outcome is the prevention coverage reached voluntarily by individuals, expressed as a function of the dynamical system's parameters. In particular, we obtained the voluntary prevention coverage as a function of prevention effectiveness and the relative cost of prevention versus treatment perceived by individuals. From a general point of view, our results suggest that epidemic elimination through the voluntary adoption of prevention is possible, even for imperfect preventive methods, provided that they are highly effective and that individuals perceive the cost of prevention relative to that of treatment being low. 

However, epidemic elimination may be only temporary. We found that the game-theoretic assumption of an equilibrium resolution of the prevention versus treatment dilemma is not ensured. In other words, there is no long-term individual motivation to adopt prevention once the epidemic is eliminated. Indeed, an important decrease in the number of infections may induce individuals to witness less disease burden (such as disease morbidity, difficulties regarding treatment adoption, disease mortality, etc.) and thus, to perceive less benefits from prevention. Therefore, epidemic elimination may induce a higher cost of prevention perceived by individuals, causing the system dynamics to return to its endemic status.

Another key outcome of our model is the effective reproduction number, which we obtained analytically. This allowed us to study it as a function of the system parameters and thus find the conditions to be met in order to ensure epidemic control (i.e., a decrease in the reproduction number) and/or elimination (i.e., reproduction number below 1). 

In the case of vaccination against childhood infectious diseases, we found that epidemic elimination required the vaccine-induced immunity to be long-lasting, in addition to high vaccine effectiveness. In the case of PrEP uptake against HIV infection, we found that the HIV epidemic may be eliminated by targeting the prevention interventions to those who identify themselves most at risk of infection. We study epidemic elimination as a function of the level of risk compensation incurred by on-PrEP high-risk MSM. In addition, we considered  alternative scenarios where individuals misperceive their risk of infection, by acknowledging only the proportion of diagnosed individuals among their peers as being infected with HIV, and where MSM dropped condom use. Our results suggest that risk misperception had a more negative impact on epidemic elimination programs, than a drop in condom use by on-PrEP MSM.

In both projects, we found that risk perception plays a major role in achieving epidemic elimination: the higher the risk perceived, the wider the area in the parameter space where epidemic elimination can be reached. That is, if the perceived risk decreases, the cost that individuals are willing to pay to adopt prevention decreases as well, regardless of the level of prevention effectiveness. In other words, if individuals do not perceive themselves as being at high enough risk of infection, they are less willing to adopt preventive methods. 

\subsection{Establishing public health policies aiming at the end of communicable diseases}

Our results give insights into the issue of voluntary prevention and epidemic dynamics in the long run. This allows to place our research within the discussion about sustainability of health behaviors and may thus be helpful for public health policies aiming at epidemic elimination \cite[]{SDG_Goal3}.  

Reaching and maintaining epidemic elimination in the long run will require active efforts to keep the cost of prevention perceived as low, as well as ensuring that individuals have access to accurate, updated epidemiological data allowing them to evaluate their risk of infection. Indeed, global immunization programs aiming to end vaccine-preventable infectious diseases have already pointed out the need of sustaining ``trust in vaccines and immunization services in communities, to increase health literacy with a focus on vaccination at all levels, and to build resilience against misinformation'' \cite[]{WHO_IA2030}; while PrEP programs have identified the need to fight uptake-related difficulties perceived by individuals \cite[]{Desai2018,Sidebottom2018}, misinformation regarding PrEP effectiveness \cite[]{Young2014,Underhill2016} and social stigma and discrimination \cite[]{Young2014,PerezFigueroa2015,Arnold2016}, to increase PrEP adoption and adherence.

Roughly speaking, our results suggest that two main strategies could be established by public health policies aiming at disease elimination. During an ongoing epidemic, increasing prevention coverage by decreasing the barriers regarding acceptability and accessibility, as well as offering information about the risk of infection and the disease and treatment burden; In the case of epidemic elimination, maintaining high levels of prevention coverage by ensuring accessibility, but also by sharing information about  the achievements of preventive programs and the epidemic severity previous to their implementation. 

In addition, our results may add new perspectives into the discussion about voluntary versus mandatory prevention, by supporting individual informed decision-making, which may join public efforts towards epidemic elimination.

\section{Limitations and perspectives}

\subsection{The complexity of modeling human behavior}
%As with all modeling studies, ours encounters its limits at representing reality in detail. 
Simple models are useful to understand epidemic dynamics by interpreting their results, while keeping some flexibility for application to other contexts. However, they may leave aside some factors reflecting the complexity of human behavior. Notably, we use a system of ordinary differential equations to model disease transmission, which may fail to take into account all individual-level heterogeneity: fixed parameters are assigned for each subpopulation to transit from one compartment to another; that is, all individual behaviors are summarized into an average behavior that is assumed to be the same for all individuals within a compartment. We tried to overcome this potential limitation in the application concerning PrEP and HIV by accounting for two subpopulations (high- and low-risk MSM), represented by additional compartments. 

In addition, we modeled decision-making as a maximization of utility, which was mainly based on individuals' perception of the relative cost and the risk of infection, which was assumed to be the same for every individual. However, individuals may perceive these factors differently and the utility function may also thus be defined to explicitly account for heterogeneity in risk and cost perception. Accounting for heterogeneity in the perception of infection risk and cost among individuals, and thus in the utility, may help better understanding dynamics in terms of population heterogeneity and developing targeted prevention programs.

\subsection{Determining and interpreting the relative cost of prevention versus treatment}
In our model, the decision-making relies not only on monetary, quantitative factors, but also on subjective factors like prevention acceptability and accessibility, as well as self-awareness regarding risk of infection, the resulting values for the total expected utility and the relative cost were read from a qualitative point of view. Therefore, it is not possible to place a specific situation in a specific point of the {\it cost-axis}: in our framework, one cannot read that a real-life strategy is ``perceived as being $x$ times more beneficial'' than another strategy. 

Nevertheless, the qualitative interpretation of our results might still offer insight about the dynamical system's behavior: from an intuitive point of view, a very low cost perceived by individuals and/or cost reduction may be easier to interpret and to aim to. Hence estimating te relative cost may not be strictly needed to increase prevention coverage. Indeed, interventions may be proposed, which intuitively increase the accessibility and affordability of prevention and thus contributing to reduce the cost perceived by individuals. Then, the reduction in cost can be indirectly appreciated by monitoring the increase in prevention coverage. If feasible, estimating the relative cost would make it possible to predict the resulting prevention coverage, depending on the intervention. 

In addition, our interpretation of the (in)stability of the disease-free equilibria may offer insight on how the dynamical system may respond to interventions and, more importantly, to maintain the objective of cost reduction. 

\subsection{Information dissemination and interpretation}

During an epidemic, individuals may also face an \textit{infodemic}\footnote{The term was first used in the context of the SARS outbreak and more recently for the COVID-19 epidemic.}, an ``excessive amount of information about a problem, which makes it difficult to identify a solution''~\cite[]{WHO_CovidInfodemic}. 
The dissemination of information about epidemics and disease burden may shape individuals' perception of their risk of infection, as well as the cost related to the available preventive and therapeutic tools. In a perception-based decision-making framework, such as our model, these perceptions impact significantly the outcomes of prevention rollout programs.

For individuals to make well-informed decisions, it is essential for individuals not only to have access to accurate and clear information, but also for them to trust these sources of information. This may be achieved through intervention from many different fronts: notably, by reducing the spreading of misinterpreted research results through mass media~\cite[]{Haneef2015}; by making scientific results broadly accessible by explaining the meaning of prevention-parameters estimations such as effectiveness ~\cite[]{Underhill2016}; by reassessing the algorithmic curation of mass-media information to prioritize the shearing of transparent and quality information~\cite[]{Lorenz2020}; by revisiting and relativizing website statistics (such as the number of readers, shares, `likes', etc.) to counteract the false perception of consensus~\cite[]{Lorenz2020}; by encouraging healthcare providers to share information and their experiences through mass media~\cite[]{Hernandez2021}; by promoting collaborations between the general population and health authorities~\cite[]{WHO_InfodemicTraining}; and by making data available through open source and open data~\cite[]{Kobayashi2021}, among others.

From the modeling perspective, as mentioned above, accounting for the heterogeneity in the perception of risk and cost of prevention may help evaluating the impact on the voluntary adoption of prevention. In addition, modeling tools such as network models may be useful to further couple the epidemic spreading and the individual-level decision-making, with information spreading~\cite[]{Chang2020}.


\subsection{Considering other behavioral models}
In our model, we considered a single-player game, where individuals act in their own interest. This can be interpreted as a game between the individuals and the public health authorities aiming at epidemic elimination. Considering multi-player games would take into account the interactions between individuals. For instance, accounting for individuals' acknowledgment of herd immunity and other players' decisions would allow to model explicitly \textit{free riders}, individuals who decide to delay or not to adopt prevention, hoping to benefit from others' preventive behaviors. In the case of vaccination, previous studies have found that high vaccination rate decreased the individual's acceptance of vaccination \cite[]{Ibuka2014}. In the case of the use of PrEP, the free-rider phenomenon could be studied by considering, for instance, the probability of undergoing condomless sex with on-PrEP individuals. Considering altruism into the model by including a component for the collective utility may allow to study if voluntary prevention coverage may reach levels that optimize the payoff at the collective level \cite[]{Shim2012}. 

%\subsection{Relying on other behavioral models}
%\rev{Psychological information-motivation-behavior models ....}

\subsection{Studying epidemics in other socio-economical settings}
Our results point to prevention cost reduction being a key public health's objective to yield epidemic elimination. Our work focuses in fighting epidemic spread within high-income settings, where the notion of cost denotes mostly non-monetary aspects, especially involving the individuals' acceptability of the preventive methods. However, in low-income settings, the consequences of infections may be far more severe, and cost-reducing policies may require to target prevention accessibility rather than prevention acceptability. 

For instance, in the case of the measles epidemiology, the mortality rates can be as high as 2\% to 15\% among children in low-income settings, and mild symptoms like diarrhea and rash can become serious complications, due to malnutrition and hemorrhages \cite[]{Sever2011}. Hence, the MMR vaccine is often well accepted in low-income countries \cite[]{Larson2016} but availability issues persist: as of August 2019, 23 countries had yet to introduce the second dose of MMR vaccine in the national vaccination schedules \cite[]{WHO_MeaslesWW2019}. 

Similarly, in the case of the HIV epidemiology, accessing HIV care in low- and middle-income settings and, in particular, the availability of PrEP might still be challenging but highly desirable given high incidence levels \cite[]{UNAIDS_Data2019}. Hence, context-specific strategies to facilitate access to PrEP should be considered and implemented \cite[]{Rebe2019}. 

\subsection{The impact of the COVID-19 pandemic on prevention interventions against measles and HIV}
The COVID-19 epidemic provoked a sanitary crisis worldwide, also impacting public health programs aiming for disease elimination. For instance, the COVID-19 epidemic impacted childhood immunization, with 37 countries delaying immunization activities during the first wave of the epidemic \cite[]{WHO_CovidMeasles}. Thus, public health authorities around the world will need to address the decrease in prevention coverage that the COVID-19 epidemic provoked during the year 2020, to maintain measles on the path of epidemic elimination. The COVID-19 epidemic also impacted HIV care and prevention accessibility, including PrEP uptake. For instance, the increase in PrEP use in France was slower during the first semester of 2020 (cf.~\figref{fig:PrEP_IdF}). A survey conducted among $\sim8\,350$ French MSM regarding their sexual behaviors during the period June--July 2020 found that 60\% of respondents had declared a complete drop in casual sexual encounters, and that 59\% of PrEP users had stopped using PrEP due to a decrease in sexual activity \cite[]{Velter2020}. 

Hence, the epidemic dynamics of the infectious diseases that are studied in this thesis are currently perturbed. The completion of our research project took place before the emergence of the COVID-19 pandemic; still, our results remain relevant, since we analyze the system at the equilibrium, in the long term, which is not impacted by perturbations (i.e., relatively small, short-term changes in epidemic dynamics). 

\section{Conclusion}
The methods developed in this doctoral research program allowed us to study infectious disease epidemics in terms of individual's attitudes towards the available preventive methods against infection, as well as their perception of the risk of infection. In particular, we focused on studying the conditions under which epidemic elimination could be reached through the voluntary participation of the target population.

We found that public health programs may yield epidemic elimination, provided that i) highly-effective preventive methods are available and perceived as being at low cost; and ii) that so individuals have a fair perception of their risk of infection. However, once the epidemic is eliminated, active efforts from public health authorities are needed to maintain the individual perception of the prevention cost low, so the willingness to keep using prevention maintains voluntary-prevention coverage at sufficiently high levels.

Our results thus suggest that individual-level risk and cost perception are essential for placing epidemics in the path towards elimination through preventive programs based on informed decision-making. Hence, the accurate and broad communication of scientific results, including epidemiological data, disease parameters estimations, and the efficacy, side effects of both preventive and therapeutic tools, as well as their impact on the epidemic, becomes a crucial focus of health programs on infectious disease prevention; especially in the current context where individuals are exposed to massive and fast spreading information. 